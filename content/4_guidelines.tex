\label{chap:guidelines}
The two "main.tex" files of this template includes 6 areas. As a user, it is not recommended to modify "Formatting Rules" or "Generic Setup", as these define the rules of the template and make it work properly.

The content of the thesis is specified in the "Content" section at the end of the "main.tex" file. The front matter and back matter should not need modified and will create the table of contents, table of figures, and bibliography. The main matter will be the most modified section and will specify the chapters, sections and subsections.

\begin{enumerate}
    \item Name and Title - Specifies the title page and authors/professors.
    \item Formatting Rules - Defines rules for the university.
    \item Generic Setup - Defines rules for general latex functioning. Necessary packages are included here.
    \item Packages - Allows installing of additional packages for the thesis.
    \item New Commands - Allows specifying shortcuts for the thesis.
    \item Content - Defines the structure and thesis sections.
\end{enumerate}


\textbf{LUH Requirements}\\
Having spoken to a few professors, there does not appear to be a university standard. Some departments are different and have there own. However, having looked at several example thesis document, the developed template seems to match several of those.
\\\\
\textbf{SPBPU Requirements}\\
A copy of the original formatting rules document\cite{Polytech2018} in russian and english (google translated) is available in the reference folder of this repository.

\label{sec:examples_algorithms}
Algorithms are not automatic. LUH prefers the algorithmic style while SPbPU requires figures. Below are examples of both.

Note: The caption of the below figure has been overridden to follow the SPbPU format. 

Warning: This changes the list of figures/algorithms as well as the counts. 

%Alg: Split range
\begin{algorithm}[H]
    \centering
    \footnotesize
    %\scriptsize
    \begin{algorithmic}[1]
        \LineComment{Global variable}
        \State $R \gets rangesList$
        \Procedure{SplitRange}{$r, value$}
            \LineComment{Create new ranges}
            \State $rLow \gets range(r.Low, value)$
            \State $rHigh \gets range(value, r.High)$
            
            \LineComment{Remove old range and add new ranges}
            \State $R.Remove(r)$
            \State $R.Add(rLow)$
            \State $R.Add(rHigh)$
        \EndProcedure
    \end{algorithmic}
    \caption{LUH Algorithm Example}
    \label{alg:luh_algorithm_example}
\end{algorithm}

%Alg: Split range
\begin{figure}[H]
    \centering
    \footnotesize
    %\scriptsize
    \fbox{\parbox{0.5\textwidth}{
        \begin{algorithmic}[1]
            \LineComment{Global variable}
            \State $R \gets rangesList$
            \Procedure{SplitRange}{$r, value$}
                \LineComment{Create new ranges}
                \State $rLow \gets range(r.Low, value)$
                \State $rHigh \gets range(value, r.High)$
                
                \LineComment{Remove old range and add new ranges}
                \State $R.Remove(r)$
                \State $R.Add(rLow)$
                \State $R.Add(rHigh)$
            \EndProcedure
        \end{algorithmic}
    }}
    %\caption{SPbPU algorithm example.}    
    %\label{fig:spbpu_algorithm_example}

    %Don't use the below caption line! This is only to show the example. Use the above caption and label lines.
    \caption*{\textit{Figure X.X - SPbPU Algorithm Example}} 
\end{figure}
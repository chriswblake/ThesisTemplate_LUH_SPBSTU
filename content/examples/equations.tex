\label{sec:examples_equations}
Equations are partly automatic. The number scheme is automatic, but the commas and period must be manually added/removed for the two versions. Table~\ref{table:equations_format} shows the important rules and the below equations provide an example.

%Table: Table format
\begin{table}[H]
    \centering
\begin{threeparttable}[H]
    \renewcommand{\arraystretch}{1.3}
    \caption{Figure format}
    \label{table:equations_format}
    \setlength\tabcolsep{5pt}
    \begin{tabular}{|l|l|l|}\hline
        \tableheader Item &\tableheader LUH &\tableheader SPBPU \\\hline

        Suffix            &None        &commas ending with period\\\hline
        Eq. Location      &Centered    &Centered\\\hline
        Number Location   &Right       &Right\\\hline
        Number Style      &(7.3)       &(7-3)\\\hline

    \end{tabular}
\end{threeparttable}
\end{table}

\begin{align}
    \label{eqn:bin_count}
    N       &= \sum 1_t \\
    \label{eqn:bin_sum}
    Sum     &= \sum v_t \\
    \label{eqn:bin_sqsum}
    SqSum   &= \sum {v_t}^2 \\
    \label{eqn:bin_avg}
    \avg    &= Sum/N \\
    \label{eqn:bin_stddev}
    \stddev  &= SqSum - 2N\avg + N\avg^2
\end{align}

\begin{align}
    \label{eqn:bin_count2}
    N       &= \sum 1_t, \\
    \label{eqn:bin_sum2}
    Sum     &= \sum v_t, \\
    \label{eqn:bin_sqsum2}
    SqSum   &= \sum {v_t}^2, \\
    \label{eqn:bin_avg2}
    \avg    &= Sum/N, \\
    \label{eqn:bin_stddev2}
    \stddev  &= SqSum - 2N\avg + N\avg^2.
\end{align}
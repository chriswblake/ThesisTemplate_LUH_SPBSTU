\label{sec:introduction_applications}
Below are three example usage scenarios for using latex templates. However, only the first will be explored during the development of this work.

%List: Example applications
\begin{enumerate}
    \item \textbf{Multiple Document Formats} – Sometimes documents required several different versions. Rather than maintain multiple copies fo the same content, content can be seperated from formatting. This enables applications like an abbridged version, normal version, commented version or large-font version.

    \item \textbf{Parrallel Work} - By seperating the content and formatting into several files, multiple people can contribute independently and at the same time. A master editor can focus on layout while different writers focus on different chapters such as theory or experimental results.

    \item \textbf{Versioning} – Sections of a book can be quickly and easily be replaced, without effecting the rest of the document. Randomization functions can also be used for selecting random content as well. Examples could be randomly sampling employee reports or randomly selecting 10 questions to create a new unique exam.
\end{enumerate}